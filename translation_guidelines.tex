\documentclass[a4paper,10pt]{article}
%\documentclass[a4paper,10pt]{scrartcl}

\usepackage[utf8]{inputenc}
\usepackage[czech]{babel}
\usepackage{fullpage}

\usepackage{amsmath}
\usepackage{amsfonts}
\usepackage{amssymb}
\usepackage{listings}
\usepackage{dtsyntax} % https://code.google.com/p/dtsyntax/
\usepackage{color}
\usepackage{array,ragged2e}
\usepackage{tabularx}



\lstset{ %
  backgroundcolor=\color{white},   % choose the background color; you must add \usepackage{color} or \usepackage{xcolor}
  basicstyle=\footnotesize,        % the size of the fonts that are used for the code
  breakatwhitespace=false,         % sets if automatic breaks should only happen at whitespace
  breaklines=true,                 % sets automatic line breaking
  captionpos=b,                    % sets the caption-position to bottom
  commentstyle=\color{mygreen},    % comment style
  deletekeywords={}   ,            % if you want to delete keywords from the given language
  escapeinside={\%*}{*)},          % if you want to add LaTeX within your code
  extendedchars=true,              % lets you use non-ASCII characters; for 8-bits encodings only, does not work with UTF-8
  frame=none,                      % disables frame around the code
  keepspaces=true,                 % keeps spaces in text, useful for keeping indentation of code (possibly needs columns=flexible)
  keywordstyle=\color{blue},       % keyword style
  language=Octave,                 % the language of the code
  morekeywords={},                 % if you want to add more keywords to the set
  numbers=left,                    % where to put the line-numbers; possible values are (none, left, right)
  numbersep=5pt,                   % how far the line-numbers are from the code
  numberstyle=\tiny\color{gray},   % the style that is used for the line-numbers
  rulecolor=\color{black},         % if not set, the frame-color may be changed on line-breaks within not-black text (e.g. comments (green here))
  showspaces=false,                % show spaces everywhere adding particular underscores; it overrides 'showstringspaces'
  showstringspaces=false,          % underline spaces within strings only
  showtabs=false,                  % show tabs within strings adding particular underscores
  stepnumber=1,                    % the step between two line-numbers. If it's 1, each line will be numbered
  stringstyle=\color{mauve},       % string literal style
  tabsize=2,                       % sets default tabsize to 2 spaces
  title=\lstname                   % show the filename of files included with \lstinputlisting; also try caption instead of title
}


\title{Python to modelica translation guidelines draft}
\author{Jáchym Barvínek}
\date{}

\pdfinfo{%
  /Title    ()
  /Author   ()
  /Creator  ()
  /Producer ()
  /Subject  ()
  /Keywords ()
}

\begin{document}
\maketitle

\paragraph{Type correspondences}
A table showing which types should be translated to which.
\begin{center}
\begin{tabular}{|l|l|}
\hline
Python type & Modelica type \\
\hline
\tt float & \tt Real \\
\tt int & \tt Integer \\
\hline
\end{tabular}
\end{center}


\paragraph{Function parameters and return values}
For each parameter in the python function, we create a corresponding {\tt input} entry in the modelica code with the same name.
Python function parameters are not strictly typed and the type cannot be inferred in general. 
We will be dealing mainly with arithmetical a numerical calculations and for this reason and sake of simplicity, we well assume
that the type is {\tt Real} in cases where it cannot be inferred.
We won't translate functions with unknown numbers of parameters now. (The {\tt *} and {\tt **} operators in python.)

Example: test\_0008

Return values: Python functions return one value of unspecified type. Thus, the type must be inferred.
(Taking the chosen parameter types into account.) We'll assume it's a {\tt Real} as long as the computation is valid.
If the return value is a local variable (but nut parameter!), we shall make a corresponding {\tt output}
entry in the translation for this variable. If it's an expression or parameter, we make a new variable.
This variable is called {\tt return\_value} in the tests, another name may be chosen but note that it's necessary to
check whether this is not also a name of a local variable or a parameter. In this case, another name must be chosen.

Examples: test\_0001, test\_0002

\paragraph{Local variables}
In modelica, there is an ambiguity, where values of local variables can be first assigned. It can be either in the
{\tt protected} or the {\tt algorithm} section. I decided to only declare the names in {\tt protected} and
assign them in {\tt algorithm}. This makes the code a bit longer, but I thought it could make the translation of the
algorithm : bit more straightforward.

Example: test\_0006

\paragraph{Side effects}
In contrast to modelica, functions in python may normally have side effects. A notable example is assignment to arrays.
I suggest to solve this by tracking which parameters might be potentially affected by this and adding extra
output value to the translated function containing the modified parameter.
In the tests, the prefix {\tt modified\_} for the variable name, but again it must be checked that the name is not used
elsewhere.
In the context outside the single function, the translator must also take this into consideration.
In the tests, the variable is tracked up to redefinition and then assigned the current value there. If it's not redefined
or there is no return value, it's tracked in the whole body of the function.

Examples: test\_0014, test\_0015

Important note. The arithmetic-assignment operators in python cause side-effects. But assignment to an arithmetic expression
does not. For example:
If {\tt a} is an array and a parameter of a function in Python which was not redefined, the code {\tt a = a + x} does not
change it's original value (but redefines values), but the code {\tt a += x} does change original and doesn't redefine locally!

Example: test\_0018


\paragraph{Array size inference}
When the translator recognizes a variable as array (or simply assumes that it's one), it needs to decide it't dimensions 
and sizes in each dimension. In Python, if a operation works on an array of n dimensions, it may also work on array 
of any higher dimension. But in basic modelica array semantics, it's not easily possible to handle arrays with dynamic
number of dimensions. Thus, the translation assumes the lowest possible dimension that makes sense according to 
arithmetical operations found in the algorithm (but at least one, zero dimension only in explicit cases like
{\tt numpy.array([])} or {\tt numpy.empty((0,))}.) If the arithmetical operations do not provide any evidence, that
the variable (or parameter) is an array, it should be assumed to be a scalar number. If sizes of the array in some
dimensions cannot be inferred from arithmetical operations, they shall be unknown (i.e. {\tt [:]} in Modelica). 
We do not know any good way to translate a function which takes as parameter an array of number of dimensions which
cannot be inferred (or lower bounded). Now, the translator should issue an error when it detects such situation.

Examples: test\_0007, test\_0013.

One important things to note is that some arithmetical operations on arrays cannot be translated in a straightforward 
way, for example, modelica does not support addition of arrays of different dimensions but in python it's possible
under certain conditions. The translator must take care of this. For example: {\tt numpy.array([[1,2],[3,4]]) + numpy.array([7,5])}
results to {\tt array([[8,7],[10,9]])}. In the translation, I solve it with a for loop. 

Examples: test\_0017, test\_0019

Note also, that in python, although {\tt numpy} arrays have fixed size, the variable that contains them 

\paragraph{Anonymous functions}
Anonymous ({\tt lambda}) functions are taken out of the surrounding function and translated as normal function with a generated name.

Example: test\_0020.

\paragraph{Lists (MetaModelica)} Lists in python can be heterogeneous, i.e. a single list can contain elements of various types.
In practice, and especially regarding numerical computations on which we focus, lists are most often homogeneous, though.
Translating heterogeneous lists would be possible in theory, but it would be very complicated and probably resulted in
swollen code. Let's assume, that lists in code we translate are homogeneous. If the type of a list cannot be inferred from
the function code, we shall assume that it's float. If there is a certainty, that the list is heterogeneous, the translator
may try another strategy (such as converting the list to tuple, but that may cause other problems) otherwise it should
issue an error.


TODO multidimensional lists, list index assignment, slicing. Problem: Some basic constructs cannot be translated using builtin MetaModelica
functions on lists.

\paragraph{Tuples (MetaModelica)} 
Tuples in python are simillar to tuples in MetaModelica, but it might be difficult to determine their type, even the length may
be uncertain. If the python code provides clues about the size and types, it's simply translated to corresponding tuple in MetaModelica.


It may be sometimes impossible to determine, if a variable is a {\tt list} / {\tt tuple} or {\tt numpy.array}. In that case,
it should be assumed to be {\tt numpy.array}.


\paragraph{}

\paragraph{Built-in functions} This table shows correspondence between builtin functions and basic language constructs used 
in the tests.
\newline
\begin{tabularx}{\textwidth}{|l|l|p{5cm}|}
\hline
Python & Modelica & Notes \\
\hline
\tt range(a,b,c) & \tt (a:c:b+1) & \\
\tt range(a,b) & \tt (a:b+1) & \\
\tt range(a) & \tt (0:a+1) & \\
\tt numpy.arange &  & same as \tt range \\
\tt for x in z: --- & \tt for x in z loop --- end for;& {\tt x}, and the block {\tt ---} must be translated. {\tt z} is a {\tt range}. \\
\tt for x in lst: --- & \tt \vtop{ \hbox{for local\_variablei in}\hbox{  (1:listLength(lst)) z loop} \hbox{x := listGet(lst, local\_variablei);} \hbox{--- end for;} } & {\tt l} is a  {\tt list}. \\
\tt while b: --- & \tt while b loop --- end while;& {\tt b} and the block {\tt ---} must be translated \\
\tt \vtop{\hbox{if b1: ---} \hbox{elseif b2: ---} \hbox{else: ---}}& \tt \vtop{\hbox{if b1 then ---}\hbox{elseif b2 then ---}\hbox{else --- end if;}} & {\strut{\tt b1}, {\tt b2} and the corresponding blocks {\tt ---} must be translated} \\
\tt numpy.ones((a,b,c,...)) & \tt ones(a,b,c,...) & \\
\tt numpy.ones([a,b,c,...]) & \tt ones(a,b,c,...) & Alternative syntax.\\
\tt numpy.zeros((a,b,c,...)) & \tt zeros(a,b,c,...) & \\
\tt numpy.zeros([a,b,c,...]) & \tt zeros(a,b,c,...) & Alternative syntax.\\
\tt numpy.empty((a,b,c,...)) &  & Array of given dimensions is only declared! \\
\tt numpy.empty([a,b,c,...]) &  & Alternative syntax.\\
\tt numpy.dot(a,b) & \tt a*b & Matrix product.  \\
\tt a.dot(b) & \tt a*b & Alternative syntax.  \\
\tt numpy.array & \tt  & array constructor \\
\tt numpy.concatenate((a,b,...), z) & \tt cat(z,a,b,...) & parameter {\tt z} may be omitted in python or given as {\tt axis=z} \\
\tt numpy.concatenate([a,b,...], z) & \tt cat(z,a,b,...) & Alternative syntax. \\
\tt m.fill(v) & \tt m := fill(v, a,b,...) & Where {\tt a,b,...} are dimensions of array {\tt m}. \\
\tt numpy.sum(x) & \tt sum(x) & \\
\tt numpy.product(x) & \tt product(x) & \\
\tt [x][y] & \tt [x,y] & Indexing (general) \\
\tt [x,y] & \tt [x,y] & Indexing ({\tt numpy} arrays only) \\
\tt :a & \tt 1:a & Array slice indexing \\
\tt a: & \tt (a+1):end & Array slice indexing \\
\tt a:b & \tt (a+1):b & Array slice indexing \\
\tt numpy.identity(n) & \tt identity(n) & \\
\tt numpy.linspace(a,b,c) & \tt linspace(a,b,c) & \\
\tt m.shape & \tt size(m) & \\
\tt l.append(e) & \tt l := listAppend(l, \{e\}) & {\tt l} is a {\tt list}. \\
\tt l1 + l2 & \tt listAppend(l1, l2) & {\tt l1},{\tt l2} are {\tt list}s. {\tt +=} Has side effect. \\ 
\tt l1.extend(l2) & \tt l1 := listAppend(l1, l2) & Has side effect. \\ 

\tt del lst[i] & \tt listDelete(lst, i) & Has side effect. \\
\tt len(lst) & \tt listLength(lst) & Assuming {\tt lst} is a {\tt list}, not a {\tt numpy.array}! \\
\tt lst[i] & \tt listGet(lst, i-1) & Also assuming {\tt i >= 0}. \\
\tt lst[-i] & \tt listGet(lst, listLength(lst)-i+1) & \\
\tt x in lst & \tt listMember(x, lst) & Different semantics of {\tt in} than in {\tt for} loop statement. \\
\hline
\end{tabularx}

\end{document}
