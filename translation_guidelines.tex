\documentclass[a4paper,10pt]{article}
%\documentclass[a4paper,10pt]{scrartcl}

\usepackage[utf8]{inputenc}
\usepackage[czech]{babel}
\usepackage{fullpage}

\usepackage{amsmath}
\usepackage{amsfonts}
\usepackage{amssymb}
\usepackage{listings}
\usepackage{dtsyntax} % https://code.google.com/p/dtsyntax/
\usepackage{color}
\usepackage{array,ragged2e}


\lstset{ %
  backgroundcolor=\color{white},   % choose the background color; you must add \usepackage{color} or \usepackage{xcolor}
  basicstyle=\footnotesize,        % the size of the fonts that are used for the code
  breakatwhitespace=false,         % sets if automatic breaks should only happen at whitespace
  breaklines=true,                 % sets automatic line breaking
  captionpos=b,                    % sets the caption-position to bottom
  commentstyle=\color{mygreen},    % comment style
  deletekeywords={}   ,            % if you want to delete keywords from the given language
  escapeinside={\%*}{*)},          % if you want to add LaTeX within your code
  extendedchars=true,              % lets you use non-ASCII characters; for 8-bits encodings only, does not work with UTF-8
  frame=none,                      % disables frame around the code
  keepspaces=true,                 % keeps spaces in text, useful for keeping indentation of code (possibly needs columns=flexible)
  keywordstyle=\color{blue},       % keyword style
  language=Octave,                 % the language of the code
  morekeywords={},                 % if you want to add more keywords to the set
  numbers=left,                    % where to put the line-numbers; possible values are (none, left, right)
  numbersep=5pt,                   % how far the line-numbers are from the code
  numberstyle=\tiny\color{gray},   % the style that is used for the line-numbers
  rulecolor=\color{black},         % if not set, the frame-color may be changed on line-breaks within not-black text (e.g. comments (green here))
  showspaces=false,                % show spaces everywhere adding particular underscores; it overrides 'showstringspaces'
  showstringspaces=false,          % underline spaces within strings only
  showtabs=false,                  % show tabs within strings adding particular underscores
  stepnumber=1,                    % the step between two line-numbers. If it's 1, each line will be numbered
  stringstyle=\color{mauve},       % string literal style
  tabsize=2,                       % sets default tabsize to 2 spaces
  title=\lstname                   % show the filename of files included with \lstinputlisting; also try caption instead of title
}


\title{Python to modelica translation guidelines draft}
\author{Jáchym Barvínek}
\date{}

\pdfinfo{%
  /Title    ()
  /Author   ()
  /Creator  ()
  /Producer ()
  /Subject  ()
  /Keywords ()
}

\begin{document}
\maketitle

\paragraph{Type correspondences}
A table showing which types should be translated to which.
\begin{center}
\begin{tabular}{|l|l|}
\hline
Python type & Modelica type \\
\hline
\tt float & \tt Real \\
\tt int & \tt Integer \\
\hline
\end{tabular}
\end{center}


\paragraph{Function parameters and return values}
For each parameter in the python function, we create a corresponding {\tt input} entry in the modelica code with the same name.
Python function parameters are not strictly typed and the type cannot be inferred in general. 
We will be dealing mainly with arithmetical a numerical calculations and for this reason and sake of simplicity, we well assume
that the type is {\tt Real} in cases where it cannot be inferred.
We won't translate functions with unknown numbers of parameters now. (The {\tt *} and {\tt **} operators in python.)

Example: test\_0008

Return values: Python functions return one value of unspecified type. Thus, the type must be inferred.
(Taking the chosen parameter types into account.) We'll assume it's a {\tt Real} as long as the computation is valid.
If the return value is a local variable (but nut parameter!), we shall make a corresponding {\tt output}
entry in the translation for this variable. If it's an expression or parameter, we make a new variable.
This variable is called {\tt return\_value} in the tests, another name may be chosen but note that it's necessary to
check whether this is not also a name of a local variable or a parameter. In this case, another name must be chosen.

Examples: test\_0001, test\_0002

\end{document}
